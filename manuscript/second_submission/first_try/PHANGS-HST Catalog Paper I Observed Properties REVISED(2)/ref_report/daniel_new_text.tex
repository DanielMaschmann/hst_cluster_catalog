previous text from daniel to referee comment number 7

However we do not really agree with the statement that the MAP is well fitted by the model track in the B-V vs V-I diagram. In fact the MAP is contracted more together and the model is still offset. We discuss this issue more precise in 4.1 and added arguments why the U-B vs V-I diagram is better suited. 


It is true that the BC03 models are not directly point on the observed distribution and we did not address this in the first version of the manuscript. 
We have updated section 4.1 and discuss this in greater detail. A discussion on the different population with different models will be also given in Lee et al. (in prep.). 
Even though we address some challenges the BC03 model faces when confronted to observations we discuss in detail, that we indeed can orientate the estimated evolutionary state of the populations, we identify in the observations with the models. 

To directly address your question if more modern models do provide better fits we would not be so sure... For instance the 5Myr hook we find in the models but no evidence in observations is not present in other models such as the CB19 models (used in Orozco-Duarte+ 2022). We state this now in the text. However, every model comes with it's own challenges and we did not find a good reason to use another model. The BC03 model in fact is well known and provides a good orientation for our purpose.

% addressing the discrepancy between models and observations
% introductory words on that
\textbf{When comparing observed properties to model predictions it is important to understand the accuracy of the model and its limitations. For instance, the \citetalias{bruzual_stellar_2003} model, we show in Figure~\ref{fig:cc_compare} only shows the stellar tracks of two metallicities without any contribution of dust and nebular emission. 
When directly comparing the model tracks and the distributions of star clusters and compact associations one can notice that the tracks do not lie directly on the observed distributions.}
% the problem with the BVI diagram
\textbf{
This is most obvious for the NUV-B-V-I and U-B-V-I diagrams whereas the B-V-I diagrams seem to show less substructure in the observations and show a better agreement between the model and the observations. However, it is important to notice that this is a misconception which originates from the lack of coverage in the blue part of the spectrum. In fact, the tracks of the \citetalias{bruzual_stellar_2003} model at ages one to five Myr and 100~Myr to 13.8~Gyr are almost aligned with the reddening vector. Furthermore, the tracks do not show much evolution between 10 and 500~Myr, making it challenging to distinguish between different stellar populations. Based on HST B-V-I observations \citet{cantiello_star_2009} find a multi-modal color distribution of star clusters where old globular clusters separate from younger star clusters. However they do not find a way to distinguish stellar populations at younger ages.
This makes clear the need for NUV and U band photometry for cluster age dating. 
The NUV band (F275W) is the shortest wavelength filter available on the HST WFC3 camera that avoids the 2175 \AA\ dust feature, while the U and B bands straddle the 4000 \AA\ break.  The combination of the NUV-U-B-V-I filters serve to break the age-extinction-metallicity degeneracy, as illustrated by the untangling of the SSP tracks in the NUV-B vs V-I (top row) and U-B vs V-I (middle row) planes, and by the separation of metal-rich and metal-poor tracks, as reflected in the segregation of the old globular cluster clump from the middle-age plume. The importance of the U-band to break the age-extinction-metallicity degeneracy was explicitly discussed in \citet{smith_young_2007}.} 

% Now go into detail of the discrapency between the models and observation 
\textbf{In comparison to the B-V-I, the NUV-B-V-I and U-B-V-I diagrams show more substructure in the observations and separate better different evolutionary stages predicted by the \citetalias{bruzual_stellar_2003} model. However, we clearly see that the model tracks are not co-spatial with the observations.
There are multiple factors in play and we discuss here the most obvious ones but for a broader discussion including different models see \citet{lee23ubvi}.}
% very young stars
\textbf{
As shown in Figure~5 of \citet{turner_phangs-hst_2021} nebular emission from star formation sites in which we find the youngest clusters (${\rm <10~Myr}$) can contribute to a reddening of the V-I color.  In combination with the effect of dust reddening which is also strongest for the youngest clusters and indicated by a reddening vector in Figure~\ref{fig:cc_compare}, one can see how the upper left population we observe in C2 clusters and C3 compact associations can be explained by the models.}
% 5 Myr is an inacuracy of the modle 
\textbf{One feature which we do not see in the observed data is the hook of the 5~Myr \citetalias{bruzual_stellar_2003} model track (solar metallicity). The fact that we do not see clusters near this position or shifted along the reddening vector which indicates that the model misses this evolutionary stage. In newer models such as discussed in \citep{orozco-duarte_synthetic_2022}, This feature is no longer present.}

% MAP and reddening? mixture of reddening and metallicites...
\textbf{When regarding ages between 100 and 500~Myr for C1 clusters we find the observed clusters almost exclusively situated to the right side of the \citetalias{bruzual_stellar_2003} model track. Even though at these ages dust from the initial star formation site is already cleared out \citep[e.g.,][]{whitmore_using_2011,hollyhead_studying_2015,hannon_h_2022,maschmann_testing_2024}, moderate dust reddening of ${\rm E(B-V) \sim 0.2~mag}$ (${\rm A_v \sim 0.6~mag}$) is still observed as quantified in \citet{thilker23sed}. This can be explained by dust within the line of sight.}
% global clusters are also not really good matched ???
\textbf{Lastly, the old globular cluster clump, we observe in the color-color distribution of C1 clusters lies exactly between the 13.8~Gyr points of the solar and the 1/50th solar metallicity \citetalias{bruzual_stellar_2003} model points. This spread can be explained by a sequence of different metallicites ranging between these two extreme points and small contribution of dust reddening.}
% conclusion why we still show the BC03 modle
\textbf{keeping the limitations of the \citetalias{bruzual_stellar_2003} model and the effects of dust-reddening in mind, it is a good reference to characterize the properties of the different regions we observe in the color-color diagrams.}
Thus,

Notes on papers:
We fully agree that the non-PHANGS references in the paper need to be augmented.  Below, we indicate some of the new references (which include all those suggested by the referee)

- reference to the comprehensive review on star clusters by Renaud+18 
%\citep{renaud_star_2018}
%In this paper points out the importance of star clusters as a probe of the evolutionary path of a galaxy. 

- references to work on star clusters in M82 by Mayya+08 and NGC3370 by Cantiello+09, both based on HST-ACS BVI photometry.

- reference to Bastian+11 on M83 with HST-WFC3 UBVI photometry
     %\citep{bastian_evidence_2011}they show that environment is a major effect on cluster desruption.
    
- reference to Silva-Villa & Larsen \citep{silva-villa_star_2011} 11 on NGC 5236, NGC 7793, NGC 1313,
and NGC 45 with 
They find no direct correlation between the star formation rate density and the fraction of stars formed in clusters. However they find a correlation between the fraction of stars formed in clusters and the total U-band luminosity. 
What does this tell us?

- reference to Larsen & Richtler 1999, 2000  on ~550 young massive clusters in 21 nearby spiral galaxies using ground based observations 
%\citep{larsen_young_1999,larsen_young_2000} They find that the age distribution does not show any significant peaks indicating a continious cluster formation in comparison to a burst of star formation. 


- references to papers on studies of star clusters based on multiple colors including those suggested by the referee: Searle+80, Girardi+95, BC03, Maraston98)

-1st paragraph of section 4.1 to acknowledge that the abundances of spiral galaxies have studied for decades and known to be around solar metalicity.

