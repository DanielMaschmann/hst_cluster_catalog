\subsection{Relation to galaxy morphology}
%\label{sssect:cc_above_ms}
%NOTE: Here is input from BCW integrating Table 4. We may want to merge this section with the next ?

%A careful examination of Figure~\ref{fig:ms} in combination with our HST imaging reveals a number of trends between the positions of the galaxies in the diagram and galaxy morphology, which we can then link back to the properties of the cluster populations in each galaxy.  

%To facilitate analysis, we combine information about key galaxy morphological features and galaxy $M_*$ and $\Delta$MS in Table~\ref{tab:DeltaMS}. The classifications of global properties in Table~\ref{tab:DeltaMS} was based on visual examination by co-author BCW of a BVI image. 

%We have checked how well our visual classifications agree with prior studies in the literature for star forming bars, global spiral structure, and flocculent star formation. We find that all 16 galaxies in which we have identified star forming bars are indeed classified as barred (11 / 16 as SB and 5 / 16 as SAB) by \citet{buta15}. For our purposes here, we distinguish between star forming bars and bars without star formation, and exclude the latter from Table~\ref{tab:DeltaMS}. The galaxies which are not marked in Table~\ref{tab:DeltaMS} which were classified as barred by Buta are NGC 4536, NGC 1087, NGC 1566, NGC 2835, NGC 6744, IC 1954, NGC 685, NGC 1433, and NGC 4548.

%A similar comparison can be made to check on our classification of spiral structure, as determined by \citet{EE87}. Here we find that 8 of the 9 galaxies in which we have identified global spiral structure are consistent with the \citet{EE87} determinations. Similarly, 9 of the 11 galaxies characterized as flocculent agree. We conclude that our classifications are in reasonably good agreement with previously established determinations.

%Starting at the top of Figure~\ref{fig:ms}, we note that several of the galaxies with the largest positive residuals are galaxies with star forming bars, such as NGC 1365, NGC 1672, NGC 4303, NGC 7496, NGC 1385, NGC 1559, and NGC 4536. Similarly, we note that most of the galaxies with the largest negative residuals are flocculent and quiescent galaxies, like NGC 4826, NGC 2775, NGC 4548, NGC 1317, NGC 4571, and NGC 4698.    
%to provide a tabular representation of corr in the main sequence diagram shown in  Figure~\ref{fig:ms}. 
%Other properties that tend to be correlated with positive $\Delta$MS are the presence of star formation at the end of the bars and the presence of global spiral arms. Galaxies with bulges tend to have negative $\Delta$MS as expected.


%Other approaches that are being examined are the use of environmental masks \citep{querejeta_stellar_2021}, and classifications based on CO morphology \citep{stuber_23}.

 Returning to the barred galaxies with large positive $\Delta$MS, an examination of the star cluster color-color diagrams in Figure~\ref{fig:ub_vi_1} shows they all have prominent middle age plumes, as expected based on Figure~\ref{fig:ms_stats}. NGC 1365 and NGC 1559 have particularly rich populations of middle age clusters, consistent with their high SFR (16.90 and 3.76 M$_{\odot}$ yr$^{-1}$ respectively), their strong spiral arm structure, and the bright central star-forming ring in the case of NGC 1365 \citep{whitmore_phangs-jwst_2023}.
It is likely that the presence of a strong bar organizes the gas to flow into regions of high density, resulting in efficient star and cluster formation.

%\subsubsection{Galaxies below the MS}\label{sssect:cc_below_ms}
% some interesting galaxies we observe
%NGC 4569 ist quenched
%In some cases, a direct link can be found between global properties of the different galaxies and the cluster population. Here, we discuss some galaxies below the MS and make a connection between their cluster population and past events. 

%A particularly striking example is NGC 4569, which experienced a ram pressure stripping event about 300 Myr ago \citep{vollmer_ngc_2004,crowl_stellar_2008,boselli_spectacular_2016} that drained the galaxy's gas reservoir and quenched its star formation as it dropped below the MS in Figure\,\ref{fig:ms}. Looking at the cluster population, we can see exactly this event in the color-diagram: the regions, we associate with stellar populations of some 100 Myr and younger, are empty.

% NGC 2775 has a strongly bimodal distribution pointing to a recent starburst
Regarding flocculent galaxies, NGC 2775 has a striking color-color distributions as it appears strongly bimodal, with a strong young cluster locus and old globular cluster clump but a weak middle age plume. This galaxy is of Type a SA(r)ab with an intermediate large bulge and a flocculent disc. Almost all class 1 clusters are situated in the bulge and class 2 clusters in the disc (Figure~\ref{fig:spatial_dist_9}. The bimodal distribution originates from the combination of a relatively dust free old central region where no recent star formation took place \citep{hogg_hot_2001}, together with flocculent star formation thought to be seeded by accreted gas \citep[i.e., from the nearby companion NGC\,2777,][]{arp_properties_1991} leading to a disk rejuvenation event. %However, this galaxy is below the MS and the here used SFR estimator is not sensitive to such a recent star forming event.

% flocculent galaxies in the Virgo cluster
Interestingly, two flocculent galaxies NGC\,4571 \citep{kennicutt_evolution_1983} and NGC\,4689 \citep{elmegreen_arm_2002} exhibit a strong YCL feature. They are adjacent in Figure~\ref{fig:ms} below the main sequence and NGC\,4689 is a member of the Virgo cluster. The galaxies are not able to sustain their star formation as they are presumed to have lost most of their gas due to their environment \citep{kenney_co_1986}, resulting in a weak middle-age plume. %NGC\,4571 was suspected to be a flocculent galaxy \citep{kennicutt_evolution_1983} but weak spiral arm structures were later found in the same work which classified NGC\,4689 as a flocculent galaxy . 
%The young cluster population which is dominating the color-color distribution in both galaxies are an indication for a lack of sustained star formation in the last $100-500 {\rm Myr}$ which would build up a middle-aged cluster population.

% more general remarks on individual color-color distributions and the need of spatial distribution discussion
%Color-color diagrams have great power to find conclusions about the stellar population. In some cases, it may be obvious which global events and mechanisms have shaped the cluster population as we find it. 
%In fact color-color diagrams can be seen as a blend together of all cluster population situated in different environments. Especially the shift due to dust attenuation makes it challenging to distinguish between different populations and to conclude on their underlying mechanism.
%To gain more precise insights, we need to look at the spatial distribution of the individual cluster groups.


%It is important to mention that the SFR values are computed by averaging over the past 100\,Myr of the modelled star formation history \citep[See also discussions in][]{kennicutt_star_2012, calzetti_star_2013, flores_velazquez_time-scales_2021}. As a consequence, the relative position of the PHANGS-HST galaxies on the MS is supposed to correlate most likely with the MAP instead of the YCL since the latter only represents a relative small fraction of the star formation history ($< 10\,{\rm Myr}$). In this section we restrict the analysis to C1 and C2 clusters.
