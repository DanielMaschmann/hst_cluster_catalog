2024 March 5

We thank the referee for their careful consideration of our paper and for a constructive report.  We provide our responses to each of the referee's points below and indicate corresponding changes to the manuscript in bold.  

We note that the catalogs were publicly released on January 29 and are now available at https://archive.stsci.edu/hlsp/phangs.  We have indicated this new data access information where needed in the text. For the final data release we excluded a small number of artifacts (201 in total) from the ML catalogs. These numbers are now updated in the text and table 3.

We also note that we changed the normalization of the histograms of Figure 2. We normalized the histograms to the highest point of the ML histograms to better compare the completeness of the human sample to the ML sample.

Furthermore, we made some cosmetic modifications to Figures 18 to 27. We now show the spatial distributions with no stretch and with the same X- and Y lengths. 

>The manuscript presents a clear and concise presentation of 
>the PHANGS-HST star cluster catalog. Specific details of the 
>data are covered in the manuscript, the results are illustrated 
>by well-designed figures, and provisions are made for data 
>access. This work is a major contribution to the field and 
>merits publication in the ApJS.

Our team appreciates the referee's positive feedback.  

>On reading the manuscript there were a few issues that merit clarification:
>General: The paper is written with a very strong focus on PHANGS. 
>This can give the unintentional impression that various techniques 
>and insights are new to PHANGS. While relevant citations are included 
>in the multiple PHANGS papers that are referenced, the inclusion of 
>key external references for techniques would improve the readers' 
>understanding of how this work fits with existing studies. A few 
>examples are included below, but a more thorough check by the authors 
>would be beneficial.

We fully agree that the non-PHANGS references in the paper need to be augmented.  Below, we indicate some of the new references (which include all those suggested by the referee).

- Studies of star clusters based on multiple colors including those suggested by the referee: Searle+80, Girardi+95, BC03, Maraston98)
- observations of star clusters in M82 by Mayya+08 and NGC3370 by Cantiello+09, both based on HST-ACS BVI photometry.
-  Bastian+11 on clusters in M83 with HST-WFC3 UBVI photometry
- Silva-Villa & Larsen 2011 (HST/ACS,WFC2- NGC 5236, NGC 7793, NGC 1313, NGC 45; 660 clusters with UBVI imaging; 1700 with BVI) 
- Larsen & Richtler 1999, 2000, ~550 young massive clusters in 21 nearby spiral galaxies using ground based UBVRIHα imaging
- Comprehensive review on star clusters by Renaud+18 
- 1st paragraph of section 4.1 to acknowledge that the abundances of spiral galaxies have studied for decades and known to be around solar metalicity.


>Specifics:
>
>1. P6-L219: "To produce the training sets, human classification was performed 
>for the brightest ∼1000 candidates in each galaxy by co-author BCW."
>
>No one is free of bias and so the selection factors adopted by BCW 
>in the training sets will inevitably introduce a bias into the final 
>machine-based catalog. Since only one classifier was used, we do not 
>have an estimate in the range of uncertainty in the cluster classifications 
>using the BCW scheme. While this approach is acceptable a more complete 
>summary of differences in classification schemes and the inevitable role 
>of selection biases should be included to fully inform the reader.
>
>As shown by Whitmore+ 2021, multiple external comparisons show good 
>agreement on the identification of class 1 clusters as compared to classes 
>2-4. In the latter cases combining the uncertainties between different 
>techniques and the ML success suggests reliability levels of only about 
>50\%. The machine learning issues, while also critical, are smaller than 
>the external differences for clusters other than class 1.
>
> P6-L230: gives a comparison with results from LEGUS. However, this comparison 
> could be misinterpreted to imply a much higher level of reliability in 
> the PHANGS cluster survey, which has not been established. The reader should 
> be explicitly warned that substantial uncertainties are a generic feature in 
> the current samples of class 2-4 clusters.

Indeed, the uncertainties in human cluster classification, and whether it is preferable to use multiple human classifiers vs. a single experienced classifier, have been discussed at length since the publication of Wei+20, our initial paper which presented a proof-of-concept study to demonstrate that deep transfer learning can be successfully used to train neural networks to classify candidate star clusters for PHANGS-HST.   

As the referee notes, comparisons of BCW classifications with classifications made by other individuals and groups have already been reported in the literature to establish the level of agreement between the classifications and possible systematic biases.  

We agree that the reader should be alerted to these issues, and have added the following text to the end of section 2.3:
"It is important to be aware that there is still significant variation in the classification of C2 and C3 objects among different studies and classifiers \citep[e.g., discussion in section 6.3.3 of][]{whitmore_star_2021}.  Part of the issue is that the characteristics of the classes have not been documented with detail much beyond the descriptions at the beginning of this section \citep[e.g., see section 2 in both][]{adamo_legacy_2017, perez_starcnet_2021}.  To help make progress, in \citet{whitmore_star_2021} we provide a full description of the methodology and criteria underlying the BCW classification scheme.  However further improvement in classification consistency still requires agreement on the criteria among a full range of experts in the field, and the development of a standardized reference set of human-labelled star clusters, as we discuss in \citet{wei_deep_2020}."


> 2. P11: Were corrections for line emission incorporated into the cluster 
> models for objects with young ages? Perhaps I missed a discussion of this 
> point which might be most relevant for the more diffuse young 
> class 3 sources.

Good point.  The SSP models shown in the color-color diagrams of this paper do not include nebular emission.  We have added a note in the first paragraph of section 4.1.

There are many choices to be made regarding the SSP model to be adopted for the SED fitting.  However, this paper is focused on analysis of the observed properties of the cluster population, in particular characterization of the YCL, MAP, and OGC.  We provide further discussion in our response to other related comments made by the referee below.




>3. P13: Following from 1. above, it would be useful to separate class
>1 and 2 clusters in Figures 3 and 4 and possibly elsewhere depending 
>on how the revised figures compare. While class 2 adds statistics, 
>this sample is weighted towards young objects and with a lower 
>classification reliability may be adding noise, as can be seen in the
>comparison in Figure 11.


Another good point, and it is indeed interesting that the Class 2's in our sample do seem to trace a distinct population in age. 

As suggested by the referee, we have included a new version of Figure 3 where we show three panels. In each panel we show the histogram of the total number of human and ML clusters and also of each class.  In this way we can compare the absolute V-band magnitude of each class relative to the entire sample.  The C2s have a larger faint population (i.e., >-8) relative to the class 1's. We add text noting the larger difference between human and ML classified objects and refer to Section 3.1 in which we explain why there will be many more younger objects at fainter magnitudes due to a combination of a lower mass-to-light ratio and the luminosity/mass function.

The C1 and C2 already have been separated in most of the other plots where it would be interesting to do so (Figures 6, 8, 10, and 11).  We opt not to generate separate panels in Figure 4 for C1 and C2s and double the size of the figure since the primary conclusion that the C2s have a more significant faint population (and thus less massive population) follows from the new Figure 3. However we provide an alternate version of Figure 4 with our re-submission to show you this version.



> 4. P13-L361: Note that the analysis of star clusters based on
> multiple colors is well established (e.g. Searle+ 1980, ApJ, 239; 
> Girardi+ 1995, A\&A, 298, Bruzual \& Charlot 2003, MNRAS, 344. etc.)
> Isn't the point that while the main features of the distributions of
> clusters in color space essentially date back to Searle+ and related
> early work on simple stellar populations, this type of analysis now
> can be refined and improved in the PHANGS project as demonstrated 
> by Lee+?

Indeed, the observed colors of star clusters and SSPs have been studied for some time in different ways, and we acknowledge that this work was not sufficiently cited in the manuscript. However, as far as we know, the main features of the observed distributions have not yet been characterized as in our paper, as it requires a very large, deep sample with homogeneous morphologicial classifications across many galaxies, which has not been available until this point.  We have modified some of the language used in the introduction of Section 4, and added references to prior work as follows:

"Hence, the distributions of star clusters in color-color diagrams \textbf{have long been} studied to gain insight into the properties and evolution of the cluster population \citep[e.g.,][]{van_den_bergh_ubv_1968, searle_classification_1980, girardi_age_1995, larsen_young_1999, chandar_luminosity_2010, adamo_legacy_2017}, 
\textbf{as well as to test SSP models \citep[e.g.,][]{bruzual_stellar_2003,vazquez_optimization_2005,maraston_evolutionary_1998}.}"

To our knowledge, this is the first time that YCL, MAP, and OGC have been defined, and demonstrated to be distinct features in the UBVI and NUV-UVI plane.   Prior to the Hubble UV initiative, it had been difficult to obtain the deep high-resolution imaging blueward of the 4000A break needed for a large sample of nearby galaxies.  In fact, as we say in Lee+2022, "previous to our program, no HST wide-field U or NUV imaging was available for 80\% of the PHANGS-HST sample, and 60\% also did not have any optical imaging with either WFC3 or ACS."  

A key difference from earlier work is that we have combined the cluster populations across many nearby galaxies on the color-color diagram.  With our large census, this then represents the cluster formation history of disk galaxies.  We believe that this represents a new way of using the diagram as a diagnostic tool.


>5. P15-Figure 6: The BC03 SSP color-color cluster tracks appear to 
>systematically miss the colors of clusters aside from the young cluster 
>plume in HST B-V vs. V-I. This calls into question the precision of later 
>statements regarding the role of extinction and separation of populations and
>associated ages in Section 4.4, especially regarding the value of the U-band.
>A reader is likely to wonder if modern cluster models provide better fits 
>(e.g., compare Figure 9 with similar figures in Orozco-Duarte+ 2022, MNRAS,
>509. 522) and how this would impact the conclusions? This issue is basic and 
>goes beyond the impact of stochasticity discussed on p18-L431. (see also 
>Comment 7)

The strength of results in section 4.4 is that they are simply a characterization of the observed 2D distribution of clusters in the UVBI color-color diagram -- it's just what we see.  The definitions of the YCL, MAP, and OGC populations are completely independent of the track.  The ages referenced in that discussion are in extremely broad bins (100-500Myr and 500-13.8 Gyr) and they are only used to point out where the uncertainties in the colors are greatest.  

As the referee is suggesting, there is a great deal of potential for using this cluster sample to test and constrain SSP models.  There are also many choices for models.  Systematic investigation of the consistency of the colors of our cluster sample with various models, including the issues raised by the referee, will be the subject of the next year's work.  Complications and uncertainties in the SSP models, in particular between 1-10 Myr, is part of the reason we have focused on reporting the observed properties of our sample in this paper, which are far more likely to stand the test of time.

>7. P19: Since the cluster model track fits in the MAP in the Figure 6 HST
>B-V - V-I diagram, then the offsets in other wavelengths cannot be solely 
>due to extinction. The issue of data offsets versus the reliability of the
>models needs to be incorporated into the discussion of extinction. Inclusion
>of SSP models with moderate extinction would be useful in helping to see how
>much of the offsets are due to reddening versus other issues.

On P19 of the submitted manuscript, the calculation of the slope of the YCL is just a fit to the observed distribution of points - it is roughly parallel to the reddening vector, and this result is also completely independent of the SSP track.  We have not fit the slope of the YCL in the B-V - V-I diagram, but it is clear that it is not aligned with the reddening vector.

In this paper, basic SSP models are shown to provide overall context for discussion of the distribution of the cluster population in color-color diagrams, and not for the determination of ages and reddenings. Again we are trying to focus on the observed features of the distribution.

Due to the many complexities involved in age determination (which include choices in the underlying assumptions in the SSP models, as the referee rightly points out), we have opted to present derivation and analysis of the physical properties in a separate paper (Paper II), which has been submitted.  We have however added text in the discussion on the consistency between SSP models and the observations, and highlight that this will be an important area of work in the near future.


6. P15-L306: The importance of U-observations in breaking the degeneracy with reddening for star clusters was explicitly discussed by Smith+ 2007, ApJL. 667 which merits a reference.

Yes in Section 4.1 we gave this article as a reference for exactly this.


8. P19-Figure 10/P20 Figure 11. The differences between the ML and human classifications are significantly offset for class 1 middle age clusters, with the ML sample potentially being significantly older (\~500 Myr vs. \~100 Myr) than the human sample. This shows the expected sensitivity of the middle-age population to the depth of the data. This issue complicates the discussions of the properties of MAP group and should be discussed.

Comments 8, 9 and 10 are related, and so are our responses to them. Thus, please consider all three answers together since some issues will be addressed in a later answer.

This difference between the color-color distributions of the human and ML sample due to depth has been addressed in Section 4.2 and Figure 8. However, it is true that the peak of MAP is located at older ages for the ML sample in comparison to the human sample. In order to make clearer to the reader that there are some differences, we added a paragraph at the end of Section 4.4. 



9. P24-Table 4: The discussion of connections to properties of host galaxies is a strong feature of this study. However, the results concerning the location of galaxies relative to the SF main sequence (DeltaMS) depends on distances to the galaxies as do properties of especially the MAP clusters (e.g., the change in median mass with distance in Figure 4 that can impact MAP ages). Adopted distances would be a useful addition to Table 4. Galaxies that have direct distance determinations from stellar markers in principle should give the most reliable results. Although Figure 15 suggests distances are not a problem, given the potential importance of this issue, the analysis of a subsample of galaxies with the best determined distances could be useful in further supporting the conclusions.

We agree that the influence of distances needs to be addressed. However, we prefer to not put the distances and the underlying methods into Table 4 since they have already been published in both Lee et al. 2023 Table 1 as well as Lee et al. 2022 Table 1. We add 2 paragraphs addressing distance-related issues to Section 5.2 and also added Figure 17 to the appendix showing the fraction in each group as a function of the galaxy distance, which shows no correlation. In this plot we highlight robust distance estimators based on standard candles by full circles and the others with empty circles.  Attached to the resubmitted manuscript we also provide a figure with the median value of absolute V-band magnitude on the x-axis. We highlight in Figure 15 the galaxies with lower distances (<15 Mpc) with full circles. These figures show that the depth of the cluster sample (and the galaxy distance) are not a driving factor for the number fraction of the MAP. This is most likely due to the fact that we are computing a relative fraction and not the total number of clusters. We add a paragraph to the end of Section 5.2 to discuss this.


10. P25-Section 5.2: The range of ages in the MAP varies rapidly with the depth of the data. Comparisons therefore should be limited to clusters chosen from the same range of ages. Are the MAP data in Figure 15 limited to clusters with ages of </\~100 Myr?

The clusters in the MAP in Figure 15 are not limited to <100 Myr.  In addition to our responses to the previous two comments, we note that we find the same correlation using the human classified and ML classified samples even though the two samples have different depths. This could be due to the fact that the star (and cluster) formation rate should be relatively constant over a dynamical timescale, which is between a few hundred and several hundred Myr and spans the difference in the depths between the human classified and ML classified samples. 
We add some discussion about this to the end of Section 5.2.

11.P26: Compliments to the authors on the informative discussion of connections with galaxy morphology in Sections 7 \& 8.

We appreciate the compliment!

12. P28-Section 9: Would it be fair to say that you are showing the continued utility of classical color-diagram analyses with an emphasis on the value added by the inclusion of U-band data? As it stands the section inadvertently could be read as if the use of color-color diagrams for analysis is an innovation of the PHANGS project.

We've modified the text in Section 9 to more clearly communicate that we are building on a foundation of prior work, and appreciate the nudge from the referee.  Also see response to comment #4.  


> 13. P 28-29: It would be useful to include a statement at the end of Section 2.4 (or possibly a footnote) along the lines of : We review issues relating to the completeness of the cluster sample and their impact on our analysis in Section 9?
> Or perhaps this critical and informative section would be more effective as a standalone Appendix that can be referenced earlier in the manuscript?

We've added a sentence at the end of Section 2.4.  It doesn't really seem like Appendix material though, so we've kept it in the main body of the paper.


> 14. P30-Sec 10: Consider pointing out that with this large sample, the association between long-term survival of clusters and compact, symmetrical structures stands out in that class 2 objects are systematically younger than the class 1 clusters.

Good point.  We have added:

\textbf{The differences in the YCL, MAP, and OGC features indicate that age distributions skew younger as the degree of cluster asymmetry and central concentration increases from C1 to C3, and are consistent with the expectation that the process of cluster dissolution should yield some correlation between age and morphology (e.g., Adamo et al. 2017, Whitmore et al. 2021, and Cook et al. 2023 and references therein). (Section 4.1)}


We also have added links to the sections related to each of the conclusions.

-----------------------

Data Editor's review :

One of our expert data editors has reviewed your initial manuscript submission and has the following suggestion(s) to help improve the data, software citation and/or overall content. Please treat this as you would a reviewer's comments and respond accordingly in your report to the science editor. Questions can be sent directly to the data editors at \href{mailto:data-editors@aas.org}{data-editors@aas.org}.

The authors indicated that they used MAST data in a question during the submission process but did not provide a DOI. I think the authors want the PHANGS-CAT HLSP as mentioned in footnote \#2. If so, the DOI for this is 10.17909/jray-9798. This DOI should be included in the revised manuscript with the AASTeX "\textbackslash{}dataset" command to mark up the DOI so it can be properly tagged in production, e.g. "a \textbackslash{}dataset[DOI: 10.17909/jray-9798]\{\href{https://urldefense.com/v3/__https://doi.org/10.17909/jray-9798__;!!CrWY41Z8OgsX0i-WU-0LuAcUu2o!x0H9flwG8xf6vpsrzrS__qjOD78S-nTxDs3LeFsBHvpsdMh0v7uanWu3OzPlbAsGTBfHkJ4MEt7RSFFpEg-31-eSaTg6$}{https://doi.org/10.17909/jray-9798}\}"