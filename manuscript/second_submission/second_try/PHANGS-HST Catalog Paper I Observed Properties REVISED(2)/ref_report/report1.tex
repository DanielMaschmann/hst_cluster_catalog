\textbf{From:} \href{mailto:lee.armus@aasjournals.org}{lee.armus@aasjournals.org} <\href{mailto:lee.armus@aasjournals.org}{lee.armus@aasjournals.org}>

\textbf{Sent:} Friday, January 12, 2024 4:28 PM

\textbf{To:} Maschmann, Daniel - (danielmaschmann) <\href{mailto:danielmaschmann@arizona.edu}{danielmaschmann@arizona.edu}>

\textbf{Cc:} \href{mailto:lee.armus@aasjournals.org}{lee.armus@aasjournals.org} <\href{mailto:lee.armus@aasjournals.org}{lee.armus@aasjournals.org}>

\textbf{Subject:} [EXT]ApJS AAS51139: Reviewer Report12-Jan-2024

Dr. Daniel Maschmann

University of Arizona

933 N Cherry Ave

Tucson, Arizona 85719

Title: PHANGS-HST catalogs for ∼100,000 star clusters and compact associations in 38 galaxies: I. Observed properties

Dear Dr. Maschmann,

I have received the reviewer's report on your above submission to The Astrophysical Journal Supplement Series, and is appended below. As you will see, the reviewer thinks that your manuscript is interesting and that it will merit publication once you have addressed the issues raised in the report.

I am also including, at the end of the report, comments from our data/software editor that I would like you to take into consideration in preparing your revision.

When you resubmit, please outline the revisions you have made in response to each of the reviewer's comments using plain text in the field provided when you upload the revised manuscript. Citing each of the reviewer's comments immediately followed by your response would be particularly helpful.

Click the link below to upload your revised manuscript;

\href{https://urldefense.com/v3/__https://aas.msubmit.net/cgi-bin/main.plex?el=A7KO7Bfq4A4GyEj3J4A9ftdDCBI5ZP2zNkzQUMDgyiNFQZ__;!!CrWY41Z8OgsX0i-WU-0LuAcUu2o!x0H9flwG8xf6vpsrzrS__qjOD78S-nTxDs3LeFsBHvpsdMh0v7uanWu3OzPlbAsGTBfHkJ4MEt7RSFFpEg-312W8n-Mp$}{https://aas.msubmit.net/cgi-bin/main.plex?el=A7KO7Bfq4A4GyEj3J4A9ftdDCBI5ZP2zNkzQUMDgyiNFQZ}

Alternatively, you may also log into your account at the EJ Press web site, \href{https://urldefense.com/v3/__https://apj.msubmit.net__;!!CrWY41Z8OgsX0i-WU-0LuAcUu2o!x0H9flwG8xf6vpsrzrS__qjOD78S-nTxDs3LeFsBHvpsdMh0v7uanWu3OzPlbAsGTBfHkJ4MEt7RSFFpEg-31yeIT13M$}{https://apj.msubmit.net}. Please use your user's login name: dmaschmann. You can then ask for a new password via the Unknown/Forgotten Password link if you have forgotten your password.

Reviewers find it helpful if the changes in the text of the manuscript are easily distinguishable from the rest of the text. We ask you to highlight the changes in bold. The highlighting can be removed easily after the review process.

The AAS Journals have adopted a policy that manuscript files become inactive, and are considered to have been withdrawn six months after the most recent reviewer's report is sent to the authors.

If you have any questions, feel free to contact me.

Sincerely,

Lee Armus

AAS Scientific Editor

California Institute of Technology

\href{mailto:lee.armus@aas.org}{lee.armus@aas.org}

----------------------------------------------------------------------

Reviewer:

Referee Report-AAS51139 Maschman+

The manuscript presents a clear and concise presentation of the PHANGS-HST star cluster catalog. Specific details of the data are covered in the manuscript, the results are illustrated by well-designed figures, and provisions are made for data access. This work is a major contribution to the field and merits publication in the ApJS.

On reading the manuscript there were a few issues that merit clarification:

General: The paper is written with a very strong focus on PHANGS. This can give the unintentional impression that various techniques and insights are new to PHANGS. While relevant citations are included in the multiple PHANGS papers that are referenced, the inclusion of key external references for techniques would improve the readers' understanding of how this work fits with existing studies. A few examples are included below, but a more thorough check by the authors would be beneficial.

Specifics:

1. P6-L219: "To produce the training sets, human classification was performed for the brightest ∼1000 candidates in each galaxy by co-author BCW."

No one is free of bias and so the selection factors adopted by BCW in the training sets will inevitably introduce a bias into the final machine-based catalog. Since only one classifier was used, we do not have an estimate in the range of uncertainty in the cluster classifications using the BCW scheme. While this approach is acceptable a more complete summary of differences in classification schemes and the inevitable role of selection biases should be included to fully inform the reader.

As shown by Whitmore+ 2021, multiple external comparisons show good agreement on the identification of class 1 clusters as compared to classes 2-4. In the latter cases combining the uncertainties between different techniques and the ML success suggests reliability levels of only about 50\%. The machine learning issues, while also critical, are smaller than the external differences for clusters other than class 1.

P6-L230: gives a comparison with results from LEGUS. However, this comparison could be misinterpreted to imply a much higher level of reliability in the PHANGS cluster survey, which has not been established. The reader should be explicitly warned that substantial uncertainties are a generic feature in the current samples of class 2-4 clusters.

2. P11: Were corrections for line emission incorporated into the cluster models for objects with young ages? Perhaps I missed a discussion of this point which might be most relevant for the more diffuse young class 3 sources.

3. P13: Following from 1. above, it would be useful to separate class 1 and 2 clusters in Figures 3 and 4 and possibly elsewhere depending on how the revised figures compare. While class 2 adds statistics, this sample is weighted towards young objects and with a lower classification reliability may be adding noise, as can be seen in the comparison in Figure 11.

4. P13-L361: Note that the analysis of star clusters based on multiple colors is well established (e.g. Searle+ 1980, ApJ, 239; Girardi+ 1995, A\&A, 298, Bruzual \& Charlot 2003, MNRAS, 344. etc.) Isn't the point that while the main features of the distributions of clusters in color space essentially date back to Searle+ and related early work on simple stellar populations, this type of analysis now can be refined and improved in the PHANGS project as demonstrated by Lee+?

5. P15-Figure 6: The BC03 SSP color-color cluster tracks appear to systematically miss the colors of clusters aside from the young cluster plume in HST B-V vs. V-I. This calls into question the precision of later statements regarding the role of extinction and separation of populations and associated ages in Section 4.4, especially regarding the value of the U-band. A reader is likely to wonder if modern cluster models provide better fits (e.g., compare Figure 9 with similar figures in Orozco-Duarte+ 2022, MNRAS, 509. 522) and how this would impact the conclusions? This issue is basic and goes beyond the impact of stochasticity discussed on p18-L431. (see also Comment 7)

6. P15-L306: The importance of U-observations in breaking the degeneracy with reddening for star clusters was explicitly discussed by Smith+ 2007, ApJL. 667 which merits a reference.

7. P19: Since the cluster model track fits in the MAP in the Figure 6 HST B-V - V-I diagram, then the offsets in other wavelengths cannot be solely due to extinction. The issue of data offsets versus the reliability of the models needs to be incorporated into the discussion of extinction. Inclusion of SSP models with moderate extinction would be useful in helping to see how much of the offsets are due to reddening versus other issues.

8. P19-Figure 10/P20 Figure 11. The differences between the ML and human classifications are significantly offset for class 1 middle age clusters, with the ML sample potentially being significantly older (\~500 Myr vs. \~100 Myr) than the human sample. This shows the expected sensitivity of the middle-age population to the depth of the data. This issue complicates the discussions of the properties of MAP group and should be discussed.

9. P24-Table 4: The discussion of connections to properties of host galaxies is a strong feature of this study. However, the results concerning the location of galaxies relative to the SF main sequence (DeltaMS) depends on distances to the galaxies as do properties of especially the MAP clusters (e.g., the change in median mass with distance in Figure 4 that can impact MAP ages). Adopted distances would be a useful addition to Table 4. Galaxies that have direct distance determinations from stellar markers in principle should give the most reliable results. Although Figure 15 suggests distances are not a problem, given the potential importance of this issue, the analysis of a subsample of galaxies with the best determined distances could be useful in further supporting the conclusions.

10. P25-Section 5.2: The range of ages in the MAP varies rapidly with the depth of the data. Comparisons therefore should be limited to clusters chosen from the same range of ages. Are the MAP data in Figure 15 limited to clusters with ages of </\~100 Myr?

11.P26: Compliments to the authors on the informative discussion of connections with galaxy morphology in Sections 7 \& 8.

12. P28-Section 9: Would it be fair to say that you are showing the continued utility of classical color-diagram analyses with an emphasis on the value added by the inclusion of U-band data? As it stands the section inadvertently could be read as if the use of color-color diagrams for analysis is an innovation of the PHANGS project.

13. P 28-29: It would be useful to include a statement at the end of Section 2.4 (or possibly a footnote) along the lines of : We review issues relating to the completeness of the cluster sample and their impact on our analysis in Section 9?

Or perhaps this critical and informative section would be more effective as a standalone Appendix that can be referenced earlier in the manuscript?

14. P30-Sec 10: Consider pointing out that with this large sample, the association between long-term survival of clusters and compact, symmetrical structures stands out in that class 2 objects are systematically younger than the class 1 clusters.

-----------------------

Data Editor's review :

One of our expert data editors has reviewed your initial manuscript submission and has the following suggestion(s) to help improve the data, software citation and/or overall content. Please treat this as you would a reviewer's comments and respond accordingly in your report to the science editor. Questions can be sent directly to the data editors at \href{mailto:data-editors@aas.org}{data-editors@aas.org}.

The authors indicated that they used MAST data in a question during the submission process but did not provide a DOI. I think the authors want the PHANGS-CAT HLSP as mentioned in footnote \#2. If so, the DOI for this is 10.17909/jray-9798. This DOI should be included in the revised manuscript with the AASTeX "\textbackslash{}dataset" command to mark up the DOI so it can be properly tagged in production, e.g. "a \textbackslash{}dataset[DOI: 10.17909/jray-9798]\{\href{https://urldefense.com/v3/__https://doi.org/10.17909/jray-9798__;!!CrWY41Z8OgsX0i-WU-0LuAcUu2o!x0H9flwG8xf6vpsrzrS__qjOD78S-nTxDs3LeFsBHvpsdMh0v7uanWu3OzPlbAsGTBfHkJ4MEt7RSFFpEg-31-eSaTg6$}{https://doi.org/10.17909/jray-9798}\}"