%%%%%
put somewhere in the spatial dist section

In the Figures~\ref{fig:color_color_regions} and ~\ref{fig:color_color_regions_nuvb}, we show the contours of all three characteristic regions for C2 clusters in the middle panels. There is a overlap between the YCL and the MAP. In order to classify clusters, we assign the clusters situated in these over lap region to the MAP, as they are most likely older than 10\,Myr. However, younger clusters can be shifted out of the YCL due to dust reddening and contaminate the MAP and OGCC. We therefore select clusters with an age of $< 10\,{\rm Myr}$ which are situated inside the MAP or OGCC and reassign them to the YCL. 
In order to include these clusters with the young cluster locus, we exclude all clusters found in the latter two regions with an age of $< 10\,{\rm Myr}$ and assign them to the young cluster locus. For this, we used the ages provided through SED fitting in \citet{thilker23sed}.

% other correlation between spatial dist of cluster and globular properties
In-depth studies of the spatial distribution of clusters can also provide insights on events which shape the host galaxies. The spatial distribution of middle age and young clusters in NGC\,4826, for instance, is heavily lopsided. This asymmetry was already observed in \citet{garcia-burillo_molecular_2003} and is most likely the result of a past gas accretion or merger event which resulted in two counter-rotating gaseous disc \citep{braun_counter-rotating_1992,rix_placid_1995}.

% completness and bright centers
From what we have seen so far is the relation between young clusters and GMC positions in good agreement on our current understanding of star formation. Nevertheless, do we find in some galaxies large central molecular gas reservoirs but a striking lack of any clusters. This is the case for NGC\,1566, 3627, 1317 and 4548 and on the HST images we can not identify strong dust obscuration. In fact, the centers of these galaxies are too bright, so that it is impossible to distinguish star clusters from the bright background. This results in an incompleteness of the star cluster population as a large part of central star cluster is missing which can for example represent a recent nuclear star burst.  
 