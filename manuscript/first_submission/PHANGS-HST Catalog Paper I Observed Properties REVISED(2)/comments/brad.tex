There is a lot of good stuff in the draft but it is still pretty disjoint. This reflects that it has been refocussed a couple times. Personally I think the focus can be more on Figure 11 (Main Sequence = MS) and similar comparisons between residuals from the MS and various properties derived from the C-C diagrams. Similarly, you could add the galaxy color - mass diagram and have similar discussion (e.g., some of the objects are probably in the "green valley").

 

Related to this, I would apologize less for various shortcomings in the catalogs (distance dependence, difference in depth, systematic differences between human and ML, ...). These will always be present (e.g., in LEGUS as well), and while we need to include them in a caveats section (just once - currently it seems like the topic comes up in 3 or 4 sections), and  warn people they need to be aware of them (and should try using both the human and ML catalogs to see if their main results are affected - this is a primary advice from Whitmore 2021), I don't think we need to repeat the warnings quite as often.

 

Another general point is that a lot of discussions have just the last 1 or 2 references, for example just Adamo 2017 and/or Krumholz (2019). In general you should try to also include the earliest reference, since that is the person that "discovered" the point. An example is that the Bastian (2012) paper was essentially the origin of the "exclusive" vs "inclusive" discussion, with Chandar and Whitmore 2014 then following up on it. I will mention some of these below. One of the good things about having the larger PHANGS community reviewing it i that they can broaden out the paper by adding  references.

 

 

%1. Intro - Detection and selection is both Thilker AND Whitmore 2021 paper. The Whitmore 2021 paper is where the exact definition of the criteria and procedure (e.g., the contrast test) was described.


%The Thilker paper is the definition of how we find the candidates, not how they were then classified.

 


%2. Into - "... using improved strategies for spectral energy distribution (SED) fit-ting"  >>> PIONEERED BY WHITMORE ET. AL. 2023a.

 

3. Sec. 2.2 or 2.3 - Discussion of Whitmore 2021. This is the paper that actually describes how the criteria for the selection of the human catalog, as mentioned in 1 above also.



%4. Sec 2.2 - pixel size is 0.04 not 0.4.

 

%5. Sec 2.3 - Much earlier discussions of cluster selection and boundness exist - e.g., Whitmore (2010), Bastian (2012), Gieles et al. 2011,  Fall and Chandar 2012, Chandar et al.  2014.

 

6. Sec 2.3 - "... the human inspection process in each galaxy began with the brightest candidates, generally proceeded to fainter objects," - Not sure where this came from. Perhaps it refers to the selection of the objects to use for the aperture corrections? In any case, it is misleading. Once the candidate list is made using Thilker et al MCI approach, I would just go through the objects in order, from bottom of field to top.


7. Sec 2.4 - C1+C2 often done, but people may/should try just C1 itself (see Whitmore 2021 who recommends this in several places).

 
8. Not sure Table 1 is very useful. Just give exp time ranges in Table 2  and people can get the exact numbers for a galaxy themselves from the data if they need it.

 

9. Keep "compact" in name for C3 in all cases. This includes the title.  Highlight that these are most similar to C8 in Multi-scale Associations, i.e., very small with only a handful of objects, but are still quite different.

 

%10. - Sec 3.1 - Whitmore et al. 2007 should be included in destruction discussion.

 

11. - While the difference between human and ML completeness is a nuisance, it is not like it would be possible to be complete anyway. All catalogs in different studies are incomplete, e.g., LEGUS - does not really go to -6 throughout. Distance, crowding, background, exp times, ... always make it difficult to compare different galaxies in a sample. >>> Need to treat ML as a different approach, with its own pros and cons and systematic differences with human. The human catalog should not be considered as just a subset of the ML. They match reasonably well (see Whitmore 2021 and Hannon 23), but there are clear systematic differences.  If selection is important for a given study, they should try using both, and see how affects results. See advice in Whitmore 2021.

 

12. Sec 4.1 - hard to read NUVBVI. better to use NUV - B vs V - I ? etc

 

13. Fig 13 -  A basic conclusion you might make explicit is that people can go to -8 relatively well before large differences show up between  ML and Human. Confusion with individual stars also become large here (e.g., whitmore 2014, i.e., HLA catalogs paper).  Nearly complete agreement at -10. 

 

14. Sec. 4.2 - "which provides evidence for the robustness of the ML classifications." >>> add "AT LEAST AT BRIGHT MAG"

 

15. MISSING SOMETHING >>> "As described in Sect. 2, we provide  ??? for each cluster in

the released catalogs with which region the cluster is associated."

 

17. Fig 4 - I agree that 5 % and 95 % (or 3 and 97 ?) might be useful, but might be good to continue with max and min as smaller symbols.

 

18. Fig 5 - Compact clusters (should also be called C3).

 

19. Have not made the point in the paper of what causes young plume (i.e., 1 - 3 My are all in the same place, so when add  reddening get a "plume").

 

20. Fig 6 - Several similar figures in Whitmore 2021 addresses similarity of ML and human in CC and when does it break down (e.g., Fig 9 - 15).

 

21 Fig. 7 - Note that in the right most-panel, the brightest peak is to the left of model. These are probably 4 - 6 My (with reddening) associations with all blue stars (i.e., no RSGs yet). This is like the left plume in stochastic models like Fouesneau et al. 2013 . Note the very linear left edge, like stars in Padova models. Should WARN people that Class 2 (probably mostly - should check) faint objects from the ML catalog may not be reliable, i.e., may be C3's)

 

22. color-diagram >> color-color diagram

 

23.  Sec 5 - Note that we are using the estimates of physical properties (i.e., < 10 My) here while we said that was going to come in paper 2. If we want to do this, we should at least acknowledge it.  "To correct for this effect, we exclude all cluster

with an age of < 10 Myr from these groups and assign them

to the young cluster group."

 

24. Sec 5 - While  I think calling out specific galaxies and correlations with spatial distributions is a good thing, it is also fairly anecdotal. It needs to be balanced by a short discussion and listing of all the many papers where more systematic correlations (e.g., 2-point correlation functions)  are discussed. Just a few that come to mind are papers by Turner 2022, Kruijssen 2012, Chevance 2020, Hwi Kim (2014), Grasha 2015,  Zhang-Fall 2001.

 

25. Sec 5. I would break this long monolithic section into several subsections with headings that help people remember what the various points are - eg., "Highest SFR", "Central Rings", " Quiescent Galaxies", 

 

26. Fig 8  - The uncertainties in the slope seems very small, i.e., different from the reddening slope a 10 sigma difference. Graphically they look very similar. Please check . Same thing for the .813 +/- .005

 

Also - do the slopes for class 2.

 

27. Fig 11 - Exactly where is the NGC 1365 data point? Same question for NGC 1512, 6744, 5068, 4689, 2775, 5248

 

 

---------------

28. I think that in many ways this figure can be the CENTERPIECE of the paper, and much more can be done with it.

 

For example, look at plots of the residuals from thiS figure with a variety of parameters taken from the  color-color diagrams such as:

 

1. Fraction of points in young plume (with different S/N cuts)

2. Fraction of points in middle age region

3. Fraction of points in old region

4. Ratio of points in young vs middle age (i.e., kind of a relative star formation measure)

5. Ratio of points in young vs old age

6. Ratio of points in middle vs old age

7. Total number of points in young, middle, old region separately

 

Also do these correlations with mass, and SFR separately - rather than the residuals in mass-SFR plane.

 

Also do this all with the color - mass plane (i.e., generally more useful for ellipticals - green. valley - star forming).

 

 

 

Predictions and fundamental causes.

 

BASIC CAUSE OF SPIRAL MAIN SEQUENCE (there must be lots of papers on this ?)

 

 

fundamental cause of correlation >>> if  cluster/star formation has been relatively constant - and the current epoch is typical >>> the more efficient SF galaxies will have made more stars/clusters  >>> more total stellar mass

 

- residuals should correlate with indicators of recent star formation (e.g., number in young region), to the degree it is representative of SFR in the last several billion years.

 

 

 

- Another class of objects are where they are because they are the quiescent/anemic/quenched galaxies - i.e., are were they are because SFR has been cut off.

 

>>> How well can we ID these using info from the C-C diagrams >>> ratio of young/old or middle/old

 

The question is whether you can ID the various types of galaxies (e.g., flocculent, stripped) based just on the numbers in the different regions in the color-color diagram. Perhaps try something more like in Sinan's paper in future?

 

-------------------------

 

 

29 - Sec 5 - I assume NGC 1317 (not 1314)  is stripped by being near the center of the Fornax cluster (i.e., central galaxy is NGC 1316). I assume there is a literature on this.

 

30. Sec 5 - There needs to be a separate graphic of NGC 1097, leading into the discussion of Dobbs and Pringle. The discreteness of the features in both the C-C and spatially on the  galaxy needs to be made more explicit with a graphic.

 

31. Sec 6.1 - "This caveat can be circumvented by

including the accuracy flag (See Sect. 2), indicating the like-

liness a cluster was correctly selected by the CNN. But this

comes at a cost of diminishing the sample size and possibly

truncating fainter clusters." - This would not help with systematic differences. Advice should be to try using both human and ML and see if anything changes.

 

32. Sec 7 - Seems like summary should be - here is a useful new dataset that can be used for a large number of different studies - here are a few examples of those studies ... - here are some caveats to keep in mind to maximize the use of the catalogs and beware of various "gotchas" - Here is a science nugget, using the distribution in the C-C to predict global (Main Sequence) parameters.

 

33. Fig 12 - 21 - I am worried that  some of the figures are probably dominated by artifacts, especially since the ML is used. Examples are the Old Globular Clusters = OGC in N4303 (not centrally located >>> implies they are mainly in the arms - I am guessing these are reddened young clusters).

 

How are we identifying OGC? Probably should not call OGC, just class 1+2, ML ? Maybe would be better using just class 1? Perhaps this will look better when use Daves ages ? Some of these comments may be relevant for Matt and Rupali's paper on OGCs?

 

34. Fig 12 - "distance to the MS" >>> residual from the MS line ?

 

35. Fig 12 - NOTE that first 4 (highest residuals from MS) are all barred galaxies >>> This might be a PRIMARY SCIENCE RESULTS OF PAPER ?

 

36. Add centers with cross in all figures, otherwise hard to make sure what are features are lined up in some cases.

 

37.  Fig 22 - " ... closest" >>> add "GMC".

 

It is interesting that old and middle age are roughly same. Should probably try this with 10 - 50 My and 50 - 500 separately. 